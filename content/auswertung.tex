\section{Auswertung}
\label{sec:Auswertung}
Zur Bestimmung der Austrittsarbeit $A_{\symup{K}}$ und des Verhältnisses $\frac{h}{e_0}$ werden zunächst die Grenzspannungen $U_{\symup{g}}$
der gemessenen Spektrallinien errechnet. Dazu werden die Wurzeln der gemessenen Stromstärken $I$ gegen die Gegenspannungen $U$ aufgetragen.
Die dazu verwendeten Werte sind in \autoref{tab:???} aufgelistet. 
\begin{figure}
  \centering
  \begin{subfigure}{0.49\textwidth}
    \centering
    \includegraphics[width=0.98\textwidth]{build/blaugruen.pdf}
    \caption{Blaugrün $\lambda = 491,6$ nm}
  \end{subfigure}
  \begin{subfigure}{0.49\textwidth}
    \centering
    \includegraphics[width=0.98\textwidth]{build/gelb.pdf}
    \caption{Gelb $\lambda = 577$ nm}
  \end{subfigure}

  \begin{subfigure}{0.49\textwidth}
    \centering
    \includegraphics[width=0.98\textwidth]{build/gruen.pdf}
    \caption{Grün $\lambda = 546$ nm}
  \end{subfigure}
  \begin{subfigure}{0.49\textwidth}
    \centering
    \includegraphics[width=0.98\textwidth]{build/rot.pdf}
    \caption{Rot $\lambda = 614,9$ nm}
  \end{subfigure}

  \begin{subfigure}{0.49\textwidth}
    \centering
    \includegraphics[width=0.98\textwidth]{build/violett1.pdf}
    \caption{Violett $\lambda = 435,8$ nm}
  \end{subfigure}
  \begin{subfigure}{0.49\textwidth}
    \centering
    \includegraphics[width=0.98\textwidth]{build/violett2.pdf}
    \caption{Violett $\lambda = 404,7$ nm}
  \end{subfigure}
  \caption{Plots der Wurzeln der Stromintensitäten $I$ gegen die Gegenspannung $U$ bei verschiedenen Wellenlängen$\lambda$ mit Ausgleichsgeraden.}
  \label{fig:farben}
\end{figure}

\begin{table}[!htp]
    \centering
    \caption{Stromstärken und zugehörige Spannungen der Spektrallinie $\lambda = 491,6$ nm.} 
    \label{tab:tabelle1}
        \begin{tabular}{c c c}
        
            \toprule
            { $I$ / A} & {$\sqrt{I}$ / $\sqrt{\symup{A}}$} & {$U$ / V} \\
            \midrule
                0,068 & 0,261 & 0,01 \\
                0,062 & 0,249 & 0,08 \\
                0,052 & 0,228 & 0,16 \\
                0,046 & 0,214 & 0,22 \\
                0,038 & 0,194 & 0,29 \\
                0,032 & 0,179 & 0,36 \\
                0,022 & 0,148 & 0,43 \\
                0,016 & 0,126 & 0,50 \\
                0,010 & 0,100 & 0,57 \\
                0,006 & 0,077 & 0,64 \\
                0     & 0 & 0,75 \\
            \bottomrule
        \end{tabular}
        
\end{table}
\begin{table}   
\centering 
 \caption{Stromstärken und zugehörige Spannungen der Spektrallinie $\lambda = 577,0$ nm.} 
        \begin{tabular}{c c c}
            \toprule
            { $I$ / A} & {$\sqrt{I}$ / $\sqrt{\symup{A}}$} & {$U$ / V} \\
            \midrule
               0,340 & 0,583 & 0,01 \\
               0,300 & 0,548 & 0,05 \\
               0,280 & 0,529 & 0,10 \\
               0,230 & 0,479 & 0,15 \\
               0,200 & 0,447 & 0,19 \\
               0,160 & 0,400 & 0,23 \\
               0,125 & 0,353 & 0,28 \\
               0,060 & 0,245 & 0,33 \\
               0,032 & 0,179 & 0,37 \\
               0,012 & 0,109 & 0,41 \\
               0     & 0 & 0,45 \\ 
            \bottomrule
        \end{tabular}
       
\end{table}
\begin{table}
\centering
\caption{Stromstärken und zugehörige Spannungen der Spektrallinie $\lambda = 546,0$ nm.} 
        \begin{tabular}{c c c}
            \toprule
            { $I$ / A} & {$\sqrt{I}$ / $\sqrt{\symup{A}}$} & {$U$ / V} \\
            \midrule
               0,940 & 0,096 & 0,01 \\
               0,820 & 0,905 & 0,06 \\
               0,740 & 0,860 & 0,11 \\
               0,660 & 0,812 & 0,16 \\
               0,580 & 0,761 & 0,22 \\
               0,440 & 0,663 & 0,27 \\
               0,340 & 0,583 & 0,32 \\
               0,225 & 0,474 & 0,38 \\
               0,135 & 0,367 & 0,43 \\
               0,050 & 0,224 & 0,50 \\
               0     & 0 & 0,57 \\
            \bottomrule
        \end{tabular}
        
        \end{table}
\begin{table}
\centering
\caption{Stromstärken und zugehörige Spannungen der Spektrallinie $\lambda = 640 $ nm.} 
        \begin{tabular}{c c c}
            \toprule
            { $I$ / A} & {$\sqrt{I}$ / $\sqrt{\symup{A}}$} & {$U$ / V} \\
            \midrule
               0,010 & 0,100 & 0,01 \\
               0,008 & 0,089 & 0,05 \\
               0,008 & 0,089 & 0,09 \\
               0,006 & 0,077 & 0,13 \\
               0,006 & 0,077 & 0,17 \\
               0,004 & 0,063 & 0,21 \\
               0,004 & 0,063 & 0,25 \\
               0,002 & 0,045 & 0,29 \\
               0,002 & 0,045 & 0,33 \\
               0,002 & 0,045 & 0,37 \\
               0 & 0& 0,41 \\
         \bottomrule
        \end{tabular}
        
        \end{table}
\begin{table}
\centering
 \caption{Stromstärken und zugehörige Spannungen der Spektrallinie $\lambda = 435,8 $ nm.} 
        \begin{tabular}{c c c}
            \toprule
            { $I$ / A} & {$\sqrt{I}$ / $\sqrt{\symup{A}}$} & {$U$ / V} \\
            \midrule
               2,000 & 1,414 & 0,01 \\
               1,750 & 1,323 & 0,10 \\
               1,550 & 1,245 & 0,20 \\
               1,300 & 1,140 & 0,30 \\
               1,150 & 1,072 & 0,40 \\
               0,850 & 0,921 & 0,50 \\
               0,650 & 0,806 & 0,60 \\
               0,380 & 0,616 & 0,70 \\
               0,205 & 0,453 & 0,80 \\
               0,075 & 0,274 & 0,90 \\
               0     & 0& 0,97 \\
         \bottomrule
        \end{tabular}
       
        \end{table}
\begin{table}
\centering
    \caption{Stromstärken und zugehörige Spannungen der Spektrallinie $\lambda = 404,7 $ nm.} 
    \label{tab:tabelle6}
        \begin{tabular}{c c c}
        
            \toprule
            { $I$ / A} & {$\sqrt{I}$ / $\sqrt{\symup{A}}$} & {$U$ / V} \\
            \midrule
            $0,650$ & $0,806$ & $0,01$ \\
            $0,600$ & $0,774$ & $0,10$ \\
            $0,520$ & $0,721$ & $0,21$ \\
            $0,460$ & $0,678$ & $0,30$ \\
            $0,380$ & $0,616$ & $0,40$ \\
            $0,320$ & $0,566$ & $0,50$ \\
            $0,260$ & $0,510$ & $0,60$ \\
            $0,190$ & $0,436$ & $0,70$ \\
            $0,125$ & $0,353$ & $0,81$ \\
            $0,060$ & $0,245$ & $0,93$ \\
            $0,020$ & $0,141$ & $1,02$ \\
            $0    $ & $0 $    & $1,19$ \\
        \bottomrule
        \end{tabular}
        
        \end{table}

\begin{figure}
    \centering
    \includegraphics[width = 0.98\textwidth]{build/plot_ug.pdf}
    \caption{Auftragung der Grenzspannung $U_\text{g}$ gegen die Frequenz $\nu$ mit Ausgleichsgeraden.}
    \label{fig:Grenzspannung}
\end{figure}    