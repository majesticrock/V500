\section{Auswertung}
\label{sec:Auswertung}
Zur Bestimmung der Austrittsarbeit $A_{\symup{K}}$ und des Verhältnisses $\frac{h}{e_0}$ werden zunächst die Grenzspannungen $U_{\symup{g}}$
der gemessenen Spektrallinien errechnet. Dazu werden die Wurzeln der gemessenen Stromstärken $I$ gegen die Gegenspannungen $U$ aufgetragen.
Die dazu verwendeten Werte sind in \autoref{tab:tabelle1} bis \autoref{tab:tabelle6} aufgelistet. Für jede der ausgemessenen Spektrallinien wird mittels Python 3.7.0 Curve Fit
eine lineare Ausgleichsrechnung der Form 
\begin{equation}
\label{eqn:ausgleich1}
  \sqrt{I} = a_{\symup{i}} \cdot U + b_{\symup{i}} 
\end{equation}
durchgeführt. Die entsprechenden Fits sind in \autoref{fig:farben} zu sehen. Die Paramteter $a$ und $b$ ergeben sich jeweils zu 
\begin{align*}
  a_1 &= (-0,30 \pm 0,01) \, \frac{\sqrt{\symup{A}}}{\symup{V}} \\
  b_1 &= (0,274 \pm 0,004) \, \sqrt{\symup{A}} \\
  a_2 &= (-1,17 \pm 0,09) \, \frac{\sqrt{\symup{A}}}{\symup{V}} \\
  b_2 &= (0,64 \pm 0,02) \,  \sqrt{\symup{A}} \\
  a_3 &= (-1,49 \pm 0,09) \, \frac{\sqrt{\symup{A}}}{\symup{V}} \\
  b_3 &= (1,03 \pm 0,02) \,  \sqrt{\symup{A}} \\
  a_4 &= (-0,16 \pm 0,01) \, \frac{\sqrt{\symup{A}}}{\symup{V}} \\
  b_4 &= (0,101 \pm 0,003) \,\sqrt{\symup{A}} \\
  a_5 &= (-1,26 \pm 0,07) \, \frac{\sqrt{\symup{A}}}{\symup{V}} \\
  b_5 &= (1,49 \pm 0,04) \,  \sqrt{\symup{A}} \\
  a_6 &= (-0,64 \pm 0,03) \, \frac{\sqrt{\symup{A}}}{\symup{V}} \\
  b_6 &= (0,86 \pm 0,02) \,  \sqrt{\symup{A}}. \\
\end{align*} 
Zu erwähnen sei hierbei, dass die Nullmessung bei jedem Fit nicht beachtet wird, das diese wie in den Abbildungen zu sehen teilweise sehr 
stark von den anderen Werten abweicht und diese Messung in der Durchführung bereits ungenau war.
Die gesuchten Grenzspannungen $U_{\symup{g, i}}$ sind die Nullstellen der Fits an der Spannungsachse. Daher ergibt sich der Zusammenhang
$U_{\symup{g, i}} = - \frac{b_{\symup{i}}}{a_{\symup{i}}}$, womit sich die Granzspannungen zu 
\begin{align*}
  U_1 &= (0,92 \pm 0,04) \, \symup{V} \\
  U_2 &= (0,54 \pm 0,04) \, \symup{V} \\
  U_3 &= (0,68 \pm 0,04) \, \symup{V} \\
  U_4 &= (0,61 \pm 0,05) \, \symup{V} \\
  U_5 &= (1,18 \pm 0,07) \, \symup{V} \\
  U_6 &= (1,34 \pm 0,07) \, \symup{V} \\
\end{align*}
ergeben.


\begin{figure}
  \centering
  \begin{subfigure}{0.49\textwidth}
    \centering
    \includegraphics[width=0.98\textwidth]{build/blaugruen.pdf}
    \caption{Blaugrün $\lambda = 491,6$ nm}
  \end{subfigure}
  \begin{subfigure}{0.49\textwidth}
    \centering
    \includegraphics[width=0.98\textwidth]{build/gelb.pdf}
    \caption{Gelb $\lambda = 577$ nm}
  \end{subfigure}

  \begin{subfigure}{0.49\textwidth}
    \centering
    \includegraphics[width=0.98\textwidth]{build/gruen.pdf}
    \caption{Grün $\lambda = 546$ nm}
  \end{subfigure}
  \begin{subfigure}{0.49\textwidth}
    \centering
    \includegraphics[width=0.98\textwidth]{build/rot.pdf}
    \caption{Rot $\lambda = 614,9$ nm}
  \end{subfigure}

  \begin{subfigure}{0.49\textwidth}
    \centering
    \includegraphics[width=0.98\textwidth]{build/violett1.pdf}
    \caption{Violett $\lambda = 435,8$ nm}
  \end{subfigure}
  \begin{subfigure}{0.49\textwidth}
    \centering
    \includegraphics[width=0.98\textwidth]{build/violett2.pdf}
    \caption{Violett $\lambda = 404,7$ nm}
  \end{subfigure}
  \caption{Plots der Wurzeln der Stromintensitäten $I$ gegen die Gegenspannung $U$ bei verschiedenen Wellenlängen $\lambda$ mit Ausgleichsgeraden.}
  \label{fig:farben}
\end{figure}
\begin{table}[!htp]
    \centering
    \caption{Stromstärken und zugehörige Spannungen der Spektrallinie $\lambda = 491,6$ nm.} 
    \label{tab:tabelle1}
        \begin{tabular}{c c c}
        
            \toprule
            { $I$ / A} & {$\sqrt{I}$ / $\sqrt{\symup{A}}$} & {$U$ / V} \\
            \midrule
                0,068 & 0,261 & 0,01 \\
                0,062 & 0,249 & 0,08 \\
                0,052 & 0,228 & 0,16 \\
                0,046 & 0,214 & 0,22 \\
                0,038 & 0,194 & 0,29 \\
                0,032 & 0,179 & 0,36 \\
                0,022 & 0,148 & 0,43 \\
                0,016 & 0,126 & 0,50 \\
                0,010 & 0,100 & 0,57 \\
                0,006 & 0,077 & 0,64 \\
                0     & 0 & 0,75 \\
            \bottomrule
        \end{tabular}
        
\end{table}
\begin{table}   
\centering 
 \caption{Stromstärken und zugehörige Spannungen der Spektrallinie $\lambda = 577,0$ nm.} 
        \begin{tabular}{c c c}
            \toprule
            { $I$ / A} & {$\sqrt{I}$ / $\sqrt{\symup{A}}$} & {$U$ / V} \\
            \midrule
               0,340 & 0,583 & 0,01 \\
               0,300 & 0,548 & 0,05 \\
               0,280 & 0,529 & 0,10 \\
               0,230 & 0,479 & 0,15 \\
               0,200 & 0,447 & 0,19 \\
               0,160 & 0,400 & 0,23 \\
               0,125 & 0,353 & 0,28 \\
               0,060 & 0,245 & 0,33 \\
               0,032 & 0,179 & 0,37 \\
               0,012 & 0,109 & 0,41 \\
               0     & 0 & 0,45 \\ 
            \bottomrule
        \end{tabular}
       
\end{table}
\begin{table}
\centering
\caption{Stromstärken und zugehörige Spannungen der Spektrallinie $\lambda = 546,0$ nm.} 
        \begin{tabular}{c c c}
            \toprule
            { $I$ / A} & {$\sqrt{I}$ / $\sqrt{\symup{A}}$} & {$U$ / V} \\
            \midrule
               0,940 & 0,096 & 0,01 \\
               0,820 & 0,905 & 0,06 \\
               0,740 & 0,860 & 0,11 \\
               0,660 & 0,812 & 0,16 \\
               0,580 & 0,761 & 0,22 \\
               0,440 & 0,663 & 0,27 \\
               0,340 & 0,583 & 0,32 \\
               0,225 & 0,474 & 0,38 \\
               0,135 & 0,367 & 0,43 \\
               0,050 & 0,224 & 0,50 \\
               0     & 0 & 0,57 \\
            \bottomrule
        \end{tabular}
        
        \end{table}
\begin{table}
\centering
\caption{Stromstärken und zugehörige Spannungen der Spektrallinie $\lambda = 640 $ nm.} 
        \begin{tabular}{c c c}
            \toprule
            { $I$ / A} & {$\sqrt{I}$ / $\sqrt{\symup{A}}$} & {$U$ / V} \\
            \midrule
               0,010 & 0,100 & 0,01 \\
               0,008 & 0,089 & 0,05 \\
               0,008 & 0,089 & 0,09 \\
               0,006 & 0,077 & 0,13 \\
               0,006 & 0,077 & 0,17 \\
               0,004 & 0,063 & 0,21 \\
               0,004 & 0,063 & 0,25 \\
               0,002 & 0,045 & 0,29 \\
               0,002 & 0,045 & 0,33 \\
               0,002 & 0,045 & 0,37 \\
               0 & 0& 0,41 \\
         \bottomrule
        \end{tabular}
        
        \end{table}
\begin{table}
\centering
 \caption{Stromstärken und zugehörige Spannungen der Spektrallinie $\lambda = 435,8 $ nm.} 
        \begin{tabular}{c c c}
            \toprule
            { $I$ / A} & {$\sqrt{I}$ / $\sqrt{\symup{A}}$} & {$U$ / V} \\
            \midrule
               2,000 & 1,414 & 0,01 \\
               1,750 & 1,323 & 0,10 \\
               1,550 & 1,245 & 0,20 \\
               1,300 & 1,140 & 0,30 \\
               1,150 & 1,072 & 0,40 \\
               0,850 & 0,921 & 0,50 \\
               0,650 & 0,806 & 0,60 \\
               0,380 & 0,616 & 0,70 \\
               0,205 & 0,453 & 0,80 \\
               0,075 & 0,274 & 0,90 \\
               0     & 0& 0,97 \\
         \bottomrule
        \end{tabular}
       
        \end{table}
\begin{table}
\centering
    \caption{Stromstärken und zugehörige Spannungen der Spektrallinie $\lambda = 404,7 $ nm.} 
    \label{tab:tabelle6}
        \begin{tabular}{c c c}
        
            \toprule
            { $I$ / A} & {$\sqrt{I}$ / $\sqrt{\symup{A}}$} & {$U$ / V} \\
            \midrule
            $0,650$ & $0,806$ & $0,01$ \\
            $0,600$ & $0,774$ & $0,10$ \\
            $0,520$ & $0,721$ & $0,21$ \\
            $0,460$ & $0,678$ & $0,30$ \\
            $0,380$ & $0,616$ & $0,40$ \\
            $0,320$ & $0,566$ & $0,50$ \\
            $0,260$ & $0,510$ & $0,60$ \\
            $0,190$ & $0,436$ & $0,70$ \\
            $0,125$ & $0,353$ & $0,81$ \\
            $0,060$ & $0,245$ & $0,93$ \\
            $0,020$ & $0,141$ & $1,02$ \\
            $0    $ & $0 $    & $1,19$ \\
        \bottomrule
        \end{tabular}
        
        \end{table}


Zur Bestimmung des Verhältnisses $\frac{h}{e_0}$, sowie der Austrittsarbeit $A_{\symup{K}}$ werden die berechneten Grenzspannungen $U_{\symup{g}}$
gegen die zugehörigen Frequenzen der Spektrallinien abgetragen. Diese sind in \autoref{tab:???} zu sehen. Die Frequenzen errechnen sich hierbei über 
\begin{equation}
\label{eqn:frequenz}
  f = \frac{c}{\lambda},
\end{equation}  
wobei $c$ die Lichtgeschwindigkeit bezeichnet. Die benötigten Wellenlängen $\lambda$ sind den Angaben and er Apparatur entnommen beziehungsweise
entstammen der jeweils angegebenen Quelle.
Dabei wird eine lineare Ausgleichsrechung der Form
\begin{equation}
  \label{eqn:ausgleich2}
  U_{\symup{g}} = \frac{h}{e_0} + A_{\symup{K}}
\end{equation}
mittels Python 3.7.0 Curve Fit durchgeführt. Die abgetragenen Spannungen und der Fit sind in \autoref{fig:Grenzspannung} zu sehen.
Die gesuchten Parameter ergeben sich damit zu 
\begin{align*}
  \frac{h}{e_0} &= (3,2 \pm 0,3) \cdot 10^{-15} \symup{eV} \\
  A_{\symup{K}} &= (-1,0 \pm 0,2) \symup{eV}. \\
\end{align*}  

 %a7 = 3.422701591615602e-15+/-3.007801772128162e-16
%b7 = -1.156772214105679+/-0.18225404834689737
\begin{figure}
    \centering
    \includegraphics[width = 0.98\textwidth]{build/plot_ug.pdf}
    \caption{Auftragung der Grenzspannung $U_\text{g}$ gegen die Frequenz $\nu$ mit Ausgleichsgerade.}
    \label{fig:Grenzspannung}
\end{figure}    

Im letzten Versuchsteil wird das Verhalten des Photostroms bei einer Bremsspannung von $-20$ V bis $20$ V an der gelben Spektrallinie
($\lambda = 577$ nm) untersucht. Die gemessenen Werte sind in \autoref{tab:messung2} aufgelistet.
\begin{table}[!htp]
\centering
\caption{Messdaten der Messung mit vergrößertem Bremsspektrum.}
\label{tab:messung2}
\begin{tabular}{c c}
\toprule
{Gegenspannung $U$ / V} & {Stromstärke $I$ / nA } \\
\midrule
19,14 & 3,200 \\
18,00 & 2,800 \\
16,00 & 3,200 \\
14,00 & 2,850 \\
12,00 & 2,600 \\
10,00 & 2,500 \\
8,00 & 2,250 \\
6,00 & 2,050 \\
4,00 & 1,600 \\
2,00 & 1,200 \\
0,00 & 0,320 \\
-2,00 & -0,040 \\
-4,00 & -0,044 \\
-6,00 & -0,046 \\
-8,00 & -0,048 \\
-10,00 & -0,050 \\
-12,00 & -0,052 \\
-14,00 & -0,048 \\
-16,00 & -0,050 \\
-18,00 & -0,048 \\
-19,13 & -0,048 \\
\bottomrule
\end{tabular}
\end{table}
Der Plot dieser Werte ist in \autoref{fig:messung2} zu sehen.
\begin{figure}
    \centering
    \includegraphics[width = 0.98\textwidth]{build/messung2.pdf}
    \caption{Auftragung des Photostroms gegen die angelegte Bremsspannung im Bremsspannungsbereich von $-20$ V bis $20$ V.}
    \label{fig:messung2}
\end{figure}
Wie zu sehen, konvergiert die Bremsspannungskurve bei hohen Beschleunigungsspannungen gegen einen Grenzwert. Dieser
kann damit erklärt werden, dass nur 
endlich viele Elektronen die Anode ereichen können, das heißt, dass beim Grenzwert annähernd alle Elektronen die Anode ereichen.
Andererseits fällt der Strom bei höheren Bremspannungen als $U_{\symup{g}}$ nicht komplett auf 0 ab; stattdessen wird ein geringer 
Gegenstrom erzeugt. Dieser lässt sich dadurch erklären, dass sich beim Kathodenmaterial schon bei Raumtemperatur Elektronen herauslösen 
und teilweise dann auch von er Anode erfasst werden können. 