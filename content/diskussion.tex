\section{Diskussion}
\label{sec:Diskussion}
Das im ersten Teil bestimmte Verhältnis 
\\ \\
\centerline{$ \frac{h}{e_0} = (3,0 \pm 0,3) \cdot 10^{-15} \, \symup{Vs} $}
\\ \\
weicht um ungefähr $27,36\%$ vom Theoriewert
\\ \\
\centerline{$ \frac{h}{e_0} = 4,13 \cdot 10^{-15} \, \symup{Vs} $}
\\ \\
ab. Trotz dieser auf den ersten Blick hoch erscheinenden Abweichung ist die Messung als gut anzusehen, da einige Fehlerquellen vorliegen.
Hierzu zählen, wie in der Auswertung bereits erwähnt, die Schwierigkeit die einzelnen Spektrallinien, welche nahe beieinander 
liegen, auseinander zu halten, da dies auf optischem Weg in diesem Aufbau nicht möglich ist. Die verwendeten Wellenlängen und die damit
bestimmten Frequenzen gelten somit nur bedingt. Hinzu kommt des Weiteren, dass die Messung der roten Spektralinie ($\lambda = 640$ nm) auf 
Grund nur sehr geringer Photoströme sehr ungenau verlief, da die Skala des verwendeten Amperemeters nicht ausreicht. Des Weiteren ist die 
Messung mit einem analogen Amperemeter ohnehin immer fehlerbehaftet, da nur mit optischer Genauigkeit abgelesen werden kann. Dies wurde zudem
zusätzlich dadurch erschwert, dass sich einige Messwerte schwer ablesen ließen, da der Photostrom teilweise leicht schwankte.
Der Fehler der Nullmessung ist in der Auswertung berücksichtigt und nimmt daher keinen Einfluss auf das Ergebnis.
Die bestimmte Austrittsarbeit
\\ \\
\centerline{$A_{\symup{K}} = (0,9 \pm 0,2) \, \symup{eV}$}
\\ \\
ist im Allgemeinen allerdings auch als gut einzuschätzen. Der Effekt welcher der Gegenstrom der Anode, wie in der Auswertung beschrieben, ist zwar 
vorhanden, allerdings als sehr gering einzustufen, weshalb der Wert der Austrittsarbeit dadurch nicht stark beeinflusst wird.